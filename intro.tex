\section{Introduction}
\subsection{Graphs and Complex networks}


\begin{frame}{Context}
	\centering
	\includesvg[inkscapearea=page,width=0.5\columnwidth]{dip}

	We wanted to study the Chilean Chamber of Deputies and its (political) polarization.\\[1em]

	In order to accomplish that, we built a model using tools from \textbf<+(1)->{\emph<1>{graph theory}} and \textbf<+(1)>{\emph<1>{statistical mechanics}}.
	
\onslide<+(1)->{%
	\vspace{3em}
	Parliament $\to$ Network  $\to$ Spin systems
}

\end{frame}


\begin{frame}[b]{Graph theory}{What is a graph?}
	\centering
	\subimport{figures/nodos-intro}{overlay}
	{\setbeamercovered{invisible}
	\begin{itemize}[<+(-3)->]
		\item Nodes and links\only<+(-3)->{, that may have direction}\only<+(-3)->{, and weights}
		%\item geography, ecology, social sciences, and many more.
\end{itemize}}
\end{frame}

%\begin{frame}{Random vs real networks}{}
%	\centering
%	\setcaptiontype{figure}
%	\subcaptionbox{Random network}{\includesvg[width=0.45\columnwidth]{random-net}}\hspace{2ex}
%	\subcaptionbox{Scale-free network}{\includesvg[width=0.45\columnwidth]{scalefree-net}}
%\end{frame}

\begin{frame}[t]{Some examples of networks}{}
	\only<1>{%
	Computer networks
	}
	\only<2>{%
	Biological networks
	}
	\only<3>{%
	Infrastructure networks
	}
	\only<4>{%
	Social networks
	}
\foreach \n in {1,...,4}{
	\only<+>{\fullsizegraphic{complex-step-\n.pdf}}
}
\end{frame}

\begin{frame}[t]{Social networks}{}
	\only<1>{\vspace{10em}\centering In these networks nodes represent people.}	
	\begin{itemize}[<+(1)->]
		\item Friendship network
		\item Criminal networks
		\item \textbf{Political networks} 
	\end{itemize}
\foreach \n in {5,...,6}{
	\only<+(-2)>{\fullsizegraphic{complex-step-\n.pdf}}
}
\end{frame}


\subsection{Legislative networks}
\begin{frame}{Legislative networks}{A specific type of political networks}
	\only<1>{\vspace{10em}\centering In these networks nodes represent legislators.}
	\foreach \n in {1,...,6}{
		        \only<+(1)>{\fullsizegraphic{papers-leg-step-\n.pdf}}
	}
	\only<+(1)->{\fullsizegraphic{papers-leg-step-7.pdf}}
	\vspace{5cm}
	\begin{itemize}[<+(1)->]
		\item Co-sponsorship to bills
		\item \textbf<+->{Coordination on roll-call votes}
	\end{itemize}
\end{frame}

\begin{frame}{Polarization}{In parliaments}
	\begin{figure}[htpb]
		%\centering
		\import{figures/wpol}{overlay}
		\label{fig:wpol}
		\vspace{12mm}
		\caption{Notion of polarization. Diagram from \citet{Neal2020}}
	\end{figure}
	\onslide<+->
	\begin{itemize}[<+->]
		\item Communities have a lot links within themselves.
		\item Few links between them.
		\item Links {\color{blue}within} communities may have weights with {\color{blue}greater} values. 
		\item<.>Links {\color{red}between} communities may have weights with {\color{red}lower} values.  \end{itemize}
\end{frame}

\begin{frame}{Polarization}{Another type}
	\vskip2ex
	\makebox[\textwidth]{
	\centering
	\includesvg[width=1.1\columnwidth]{polarization}
	\vspace{33mm}		
	}
	\captionof{table}{Different types of polarization. From~\citet{Neal2020}.\label{fig:polarization}}
	\begin{itemize}[<+->]
	\item The author proposes a new type of polarization using \textbf{signed links}.
	\item The author  suggest ideas from \textbf{balance theory} for analyzing strong polarization.
\end{itemize}
\end{frame}

\subsection{Balance theory}
\begin{frame}[t]{Balance theory}{A brief description}
	\setbeamercovered{invisible}
	\begin{itemize}
		\item Is a theory consolidated by \citet{CartwrightHarary1956}, from psychology.
		\item Studies transitive relations between individuals.
	\end{itemize}
	%\subimport{figures/friends}{overlay}
	\begin{minipage}{0.40\textwidth}
		\phantom{a}	
	\end{minipage}
	\hspace{3mm} \begin{minipage}{0.46\textwidth}
	\vspace{3em}
	\onslide<-6>{
	\begin{itemize}[<+(+2)->]
		\item If all three people are friend, there is \textbf{no conflict} 
		\item If all three people are enemies, there is \textbf{conflict} 
		\item If two people are enemies from the third, there is \textbf{no conflict} \item If only two people are enemies, there is \textbf{conflict} 
\end{itemize}}
	\end{minipage}
	\foreach \n in {1,...,7}{
		        \only<+(-3)>{\fullsizegraphic{friends-step-\n.pdf}}}

\end{frame}

\section{Summary}
\begin{frame}[t]{Summary}{Motivation and aims of our work}
	{\centering
	\includesvg[inkscapearea=page,width=0.75\columnwidth]{dip}}
	\begin{itemize}[<+->]
	\item We want to study \emph{polarization} in parliaments using tools from graph theory.
	\item Our networks will be built from roll-call votes 
	\item Inspired by the ideas from \citet{Neal2020}, the built networks will have with signed links, and also we will employ elements of balance theory.
	\item<+-> Evaluate the model with real data.
\end{itemize}
\end{frame}



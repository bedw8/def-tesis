
\subsection{Spin system}

\begin{frame}[t]{Spin systems}
	\only<1>{\vspace{10em}\centering Turn for the physics!} \only<2->{
		We are going to construct two spin systems.\\[1em]

		In one of them the links signs will act as spins, while in the other the triangles balance will play that role.\\[3em]
		\noindent\begin{columns}
			\column{0.2\textwidth}{
				\begin{itemize}
					\item Links:
				\end{itemize}
				\vspace{5em}
				\begin{itemize}
					\item Triangles:
				\end{itemize}
			}
			\column{0.7\textwidth}{
			}
		\end{columns}
		\foreach \n in {1,...,2}{
			\only<+(1)>{\fullsizegraphic{spins-step-\n.pdf}}
		}
	}


\end{frame}

\subsection{Counting spins}
\begin{frame}{Counting spins}
	We are going to count the amount of links and triangles.
\begin{itemize}
	\item Allow us to measure the probability of choosing a negative sign link or a balanced triangle randomly on a given network
	\item It will let us construct physical quantities
\end{itemize}	

\onslide<+(1)->{%
Since the network in model is fully connected, then
\begin{itemize}
	\item Amount of links:
 \begin{equation}
     L = \binom N 2 = \frac{N(N-1)}{2}
 \end{equation}   
	\item Amount of triangles:
\begin{equation}
    C = \binom N 3 = \frac{N(N-1)(N-2)}{6}
\end{equation}
\end{itemize}}
\onslide<+(1)->
We still want to count each kind of links (unsigned and signed) and each kind of triangle (balanced and unbalanced).
\end{frame}

\newcommand{\tikzmark}[1]{\tikz[baseline,remember picture] \coordinate (#1) {};}
\begin{frame}[b]{Counting links}
\setbeamercovered{invisible}
\begin{itemize}
	\item[] 
	\item[] 
	\item[] 
	\item[] 
	\item[] 
	\item[] 
	\item[] 
	\item[] 
	\item[] 
	\item[]
	\item[]
	\item<+(1)-> Positive links \begin{equation}
		L^{+} = \frac{1}{2}\sum_i^k n_i^2 - \frac{1}{2}\sum_i^{k} {n_i} =  \frac{1}{2} \|\vec n\|^2 - \frac{N}{2}. \label{eq:Lplus} \end{equation}
		\item<+(1)-> Negative links 
			\begin{equation}
L^{-} = \tikzmark{bsum}\sum_{i<j}^k n_in_j\tikzmark{asum} = \frac{N^{2}}{2} - \frac{1}{2}\|\vec n\|^2 \label{eq:Lminus} 
\end{equation}
\end{itemize}

\def\keeplast{\ifnum\n<3 \onslide<+(-2)> \else \onslide<+(-2)->\fi}
\foreach \n in {1,...,3}{
	\keeplast{%
	\begin{tikzpicture}[remember picture, overlay]
		\node[at=(current page.center),anchor=center,shift=({-13mm,7mm})] {%
				\includegraphics[width=0.9\columnwidth]{count-links-step-\n.pdf}
		};
	\end{tikzpicture}
	}
}

\foreach \i in {1,...,1}{%
\pgfmathtruncatemacro\z{\i+4}%
\onslide<+(-2)>{%
\begin{tikzpicture}[remember picture, overlay]
	\node[at=(current page.east),
		anchor=east,
		shift=({-12mm,17mm})] (matrix) {%
        \includegraphics{svg/nxn-step-\i.pdf} };
	\node[at=(matrix.north),shift=({0cm,2mm})] {$\vec{n}\otimes \vec{n}$};
\end{tikzpicture}	
	}
}

\begin{tikzpicture}[remember picture, overlay]
	\node[red,draw, 
		  at=(bsum.west),
		  anchor=west,
		  inner sep=6.2mm,
		  outer sep=1mm,
		  shift=({-1mm,0.5mm}),
		  visible on=<+(-3)>] (highlight) {};	
	  \draw[-stealth,red,visible on=<.(-3)>] (highlight.south east) to[out=-25,in=-90,distance=3cm] (matrix.south);
\end{tikzpicture}

\end{frame}

\begin{frame}[b]{Counting triangles}{Type $+++$} \foreach \n in {1,...,2}{
        \only<\n>{\fullsizegraphic{count-t+++-step-\n.pdf}}
}

	Similar to positive links, this type of triangles are only present inside of the groups.%
\begin{align}
	C_{+++} &=
\sum_a^k n_a(n_a -1)(n_a -2)/6\nonumber\\
&=  \frac{1}{6} \bigg[\|\vec n\|^3_3-3\|\vec n\|^{2} + 2N\bigg], \label{eq:C+++}
\end{align}
\end{frame}

\begin{frame}[b]{Counting triangles}{Type $---$}
\foreach \n in {1,...,2}{
        \only<\n>{\fullsizegraphic{count-t----step-\n.pdf}}
    }

These triangles will be formed by three communities putting one node each one. 
\begin{align}
	C_{---} &=
\sum_{a<b<c}^k n_a n_b n_c =
\frac{1}{6}\bigg[N^{3} + 2\|\vec n\|^{3}_{3} - 3N \|\vec n\|^{2} \bigg]. \label{eq:C---}
\end{align}
\end{frame}

\begin{frame}[b]{Counting triangles}{Type $+--$}
 \foreach \n in {1,...,2}{
	         \only<\n>{\fullsizegraphic{count-t+---step-\n.pdf}}
 }

Finally, the remaining type of triangle is counted as follows
\begin{align}
	C_{+--} &= \sum_{i<j}^{k} \frac{n_in_j(n_j-1) + n_jn_i(n_i-1)}{2}\nonumber \\
&= \frac{1}{2}\bigg[(N+1)\|\vec n\|^{2}-\|\vec n\|_{3}^{3}-N^{2} \bigg] \label{eq:C+--}
\end{align}
\end{frame}

\begin{frame}{State probabilities}
	We can define the probabilities of finding a spin in a particular state, on a given network.\\[1em]

First, we will have that 
\begin{equation}
	p_{\ell}^{+} = \frac{L^{+}}{L},\qquad \text{and}\qquad
	p_{\ell}^{-} = \frac{L^{-}}{L}
\end{equation}
are the probabilities for positive and negative links, respectively.\\[1em]

And also, 
\begin{equation}
	p_{t}^{+} = \frac{C_{+++} + C_{+--}}{C},\qquad \text{and}\qquad
	p_{t}^{-} = \frac{C_{---}}{C}
\end{equation}
are the probabilities for balanced and unbalanced triangles.

\end{frame}




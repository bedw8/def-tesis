\appendix
\section{Appendix}

\begin{frame}[allowframebreaks]{Appendix}
Recalling that $M_\ell = \|\vec{n}\|^2  - \frac{N(N+1)}{2}$, and differentiating it in order to find its critical point, we get 
$$\frac{\p}{\p n_i} \|\vec{n}\| =0,$$
which we already pointed out it yields to $\vec{n} = \{N / k, \ldots, N / k \}$.

\framebreak

 \begin{figure}
     \centering
	 \includegraphics[height=0.7\textheight]{se_me}
     \caption{Link entropy against link magnetization. Orange line is $s_\ell$ from \cref{eq:s_norms} while blue dots represent configurations from \Cref{fig:ps1}}
     \label{fig:sL_mL}
 \end{figure}
 \begin{figure}[!h]
     \centering
	 \includegraphics[height=0.7\textheight]{st_mt}
     \caption{Triangle entropy against triangle magnetization. Orange line is $s_t$ from \cref{eq:s_norms} while blue dots represent configurations from \Cref{fig:ps1}}
     \label{fig:sT_mT}
 \end{figure}

\framebreak

\begin{figure}
    \centering
    \includegraphics[height=0.7\textheight]{mL_mT}
    \caption{Triangle magnetization with respect to link magnetization. We have used the same label style for the relevant vote scenarios as in \Cref{fig:ps1}, to illustrate the map between the configuration space and the $m_t-m_\ell$ space. Values were computed for $N=155$.\label{fig:mL_mT}} 
\end{figure}    

\framebreak
\begin{figure}
    \centering
    \includegraphics[height=0.7\textheight]{mL_mT_paths}
    \caption{Same as in \Cref{fig:mL_mT}, with paths $A,B,C$ and $D$ marked up on it. We followed the same graphic style for paths from \Cref{fig:ps_paths} for easier comparison. Path $D$ was drawn with $\alpha=5$.\label{fig:mL_mT_paths}}
\end{figure}
\framebreak
\begin{equation}
    m_{t} =  \frac{3 \alpha m_{\ell}}{N - 2} + \frac{3\alpha\big[N(1-N)+4\alpha(N-\alpha)\big]}{N(N-1)(N-2)} + 1.
\label{eq:mT_of_mL}
\end{equation}
The equation for path $A$ is
\begin{equation}
    \begin{split}
        f(m_\ell) = 8N^2 - 9(N-1)(Nm_\ell -2)- (3m_\ell(N-1) + 3+N)^{3 / 2}
    \end{split}
    \label{eq:rs}
\end{equation}
\framebreak
\begin{figure}[h]
    \centering
	\includegraphics[height=0.5\textwidth]{ts_ML}
    \caption{Fluctuation of the link magnetization. Top panel: raw series; lower panel: average every 50 consecutive data points. Red lines represent the mean value for each plot.}
    \label{fig:ts1}
\end{figure}
\framebreak
\begin{figure}[h]
    \centering
    \includegraphics[height=0.5\textwidth]{ts_MT}
    \caption{Fluctuation of the triangle magnetization. Top panel: raw series; lower panel: average every 50 consecutive data points. Red lines represent the mean value for each plot.}
    \label{fig:ts2}
\end{figure}
\framebreak
\begin{figure}[h]
    \centering
    \includegraphics[height=0.5\textwidth]{ts_SL}
    \caption{Fluctuation of the link entropy (computed with $\k=1$). Top panel: raw series; lower panel: average every 50 consecutive data points. Red lines represent the mean value for each plot.}
    \label{fig:ts3}
\end{figure}
\framebreak
\begin{figure}[h]
    \centering
    \includegraphics[height=0.5\textwidth]{ts_ST}
    \caption{Fluctuation of the triangle magnetization (computed with $\k=1$). Top panel: raw series; lower panel: average every 50 consecutive data points. Red lines represent the mean value for each plot.}
    \label{fig:ts4}
\end{figure}
\framebreak
\begin{figure}[h]
    \centering
    \includegraphics[height=0.5\textwidth]{ts_T}
    \caption{Fluctuation of the T (computed with $\k=1$). Top panel: raw series; lower panel: average every 50 consecutive data points. Red lines represent the mean value for each plot.}
    \label{fig:ts5}
\end{figure}

\end{frame}




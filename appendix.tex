\appendix
\section{Appendix}


\begin{frame}{Critical points of link entropy}
\setlength\fboxrule{0.5pt}
\setlength\fboxsep{4pt}
\fbox{\begin{minipage}{0.6\textwidth}
		Let's recall that
		$$s_\ell =  - \Big[ p_\ell^+\ln(p_\ell^+) + p_\ell^-\ln(p_\ell^-) \Big],$$
		$$		p_\ell^{+} = \frac{L^{+}}{L},\qquad\text{and}\qquad		L^{+} = \left( \frac{1}{2}\|\vec{n}\|^2 - \frac{N}{2}\right)$$
\end{minipage}}\\[1em]

We differentiate $s_\ell$ to get its critical points.
\begin{align}
	\frac{\p s_\ell }{\p n_i} &= \frac{\p s_\ell}{\p p^{+}} \frac{\p p^{+}}{\p n_i} = - \bigg[ \ln(p^{+}) - \ln(p^{-}) \bigg] \left[ \frac{\|\vec{n}\|}{L}  \frac{\p}{\p n_i} \|\vec{n}\|  \right] 
.\end{align}
\\[2em]

\begin{itemize}
	\item $\left[ \frac{\|\vec{n}\|}{L}  \frac{\p}{\p n_i} \|\vec{n}\|  \right] =0\qquad  \to \qquad \vec{n} = \{N / k, \ldots, N / k \}$ 
	\item $\Big[ \ln(p^{+}) - \ln(p^{-}) \Big] = 0 \qquad \to \qquad L^{+} = L^{-}$
\end{itemize}
\end{frame}

\begin{frame}{Maximum of link entropy}
We can get an expression for the points where $L^{+}=L^{-}$.
\begin{align*}
     0 &= L^{+} - L^{-} =  2 \left[ \sum_{i=1}^{k-1} n_i^2 + \left( N-\sum_{i=1}^{k-1} n_i \right)^2   \right]   - N(N +  1) 
.\end{align*}
For the specific case of $k=3$, this results on the curve\\
\begin{equation}
	\begin{aligned}
		\frac{N \left(N - 1\right)}{4} &=  N \left(n_{1} + n_{2}\right) - n_{1} n_{2}  - n_{1}^{2} - n_{2}^{2}, \label{eq:Lp_eq_Lm}
	\end{aligned}
\end{equation}
\end{frame}

\begin{frame}{Minimum of link magnetization}
Recalling that $M_\ell = \|\vec{n}\|^2  - \frac{N(N+1)}{2}$, and differentiating it in order to find its critical point, we get 
$$\frac{\p}{\p n_i} \|\vec{n}\| =0,$$
which we already pointed out it yields to $\vec{n} = \{N / k, \ldots, N / k \}$.
\end{frame}

\begin{frame}{Link states probabilities in terms of link magnetization}
Both type of  entropies can be written in terms of their respective magnetization, but here we will focus on the link entropy.\\[1em]

Since
\begin{align*}
	m_\ell &= p_{\ell}^{+} - p_{\ell}^{-},\qquad  p_{\ell}^{+} + p_{\ell}^{-} = 1\\
	%m_t &= p_{t}^{+} - p_{t}^{-},\qquad  p_{t}^{+} + p_{t}^{-} = 1,
\end{align*}
we can express the probabilities $p_{\ell}$ as
\begin{align*}
    p_\ell^{+} &= \frac{1}{2}(1+m_\ell), \quad p_\ell^{-} = \frac{1}{2}(1-m_\ell),%\label{eq:pl+-_m}\\
	%\\p_t^{+} &= \frac{1}{2}(1+m_t),\quad
%    p_t^{-} = \frac{1}{2}(1-m_t).%\label{eq:pt+-_m}
\end{align*}
\end{frame}

\begin{frame}
 \begin{figure}
    \centering
	\fbox{\adjincludegraphics[trim={{0.13\width} {0.13\height} {0.07\width} {0.18\height}},clip,height=0.7\textheight]{Tlog}}%
	\caption{Logarithm of temperature's absolute value on top of the configuration space.}
	\label{fig:Tlog}

 \end{figure}
\end{frame}

\begin{frame}
\begin{equation}
        s_\ell = - \, \Biggr[\left(\frac{1+m_\ell}{2}\right) \ln\left(\frac{1+m_\ell}{2}\right) + \left(\frac{1-m_\ell}{2}\right)\ln \left(\frac{1-m_\ell}{2}\right)\Biggr]\label{eq:sl_m}
,\end{equation}
and
\begin{equation}
        s_t = - \, \Biggr[\left(\frac{1+m_t}{2}\right) \ln\left(\frac{1+m_t}{2}\right) + \left(\frac{1-m_t}{2}\right)\ln \left(\frac{1-m_t}{2}\right)\Biggr],\label{eq:st_m}
\end{equation}
\end{frame}

\begin{frame}
 \begin{figure}
     \centering
	 \includegraphics[height=0.7\textheight]{se_me}
     \caption{Link entropy against link magnetization. Orange line is $s_\ell$ from \cref{eq:sl_m} while blue dots represent configurations from \cref{fig:cf}}
     \label{fig:sL_mL}
 \end{figure}
\end{frame}

\begin{frame}
 \begin{figure}[!h]
     \centering
	 \includegraphics[height=0.7\textheight]{st_mt}
     \caption{Triangle entropy against triangle magnetization. Orange line is $s_t$ from \cref{eq:st_m} while blue dots represent configurations from \cref{fig:cf}}
     \label{fig:sT_mT}
 \end{figure}
\end{frame}

\subsection{Relating magnetizations}

\begin{frame}{Relating magnetizations}
	\begin{figure}
    \centering
    \includegraphics[height=0.7\textheight]{mL_mT}
    \caption{Triangle magnetization with respect to link magnetization. We have used the same label style for the relevant vote scenarios as in \Cref{fig:cf}, to illustrate the map between the configuration space and the $m_t-m_\ell$ space. Values were computed for $N=155$.\label{fig:mL_mT}} 
\end{figure}    
\end{frame}

\begin{frame}{Relating magnetizations}
	\begin{equation}A(t) =  \left(\frac{N}{3} - t, \frac{N}{3} - t, \frac{N}{3} +t\right), \label{eq:pathA}\end{equation}
	\begin{equation}B(t) = \left(\frac{N}{3} -2t,\frac{N}{3}+t,\frac{N}{3}+t\right).\label{eq:pathB}\end{equation}
	\begin{equation}C(t) = \left(0,\frac{N}{2}-t,\frac{N}{2}+t\right).\label{eq:pathC}\end{equation}
	   \begin{equation}
		   D(t,\alpha) = \left(\alpha,\frac{N}{2}-\frac{\alpha}{2}-t,\frac{N}{2}-\frac{\alpha}{2}+t\right).\label{eq:pathD}
	   \end{equation}
\end{frame}

\begin{frame}{Relating magnetizations}
\begin{figure}[!h]
    \centering
    \adjincludegraphics[trim={{0.16\width} {0.12\height} {0.17\width} {0.12\height}},clip,height=0.7\textheight]{ps_paths}
    \caption{Configuration space with the same colored circles as in \Cref{fig:cf}, and with paths between these drawn. Only 1/6 of the triangle was marked up, to avoid redundancy due to the symmetry across the triangle. Paths $A, B, C$ and $D$ are defined in \cref{eq:pathA,eq:pathB,eq:pathC,eq:pathD}. The path $D$ was drawn with $\alpha = 5$.}
    \label{fig:ps_paths}
\end{figure}
\end{frame}

\begin{frame}{Relating magnetizations}
\begin{figure}
    \centering
    \includegraphics[height=0.7\textheight]{mL_mT_paths}
    \caption{Same as in \Cref{fig:mL_mT}, with paths $A,B,C$ and $D$ marked up on it. We followed the same graphic style for paths from \Cref{fig:ps_paths} for easier comparison. Path $D$ was drawn with $\alpha=5$.\label{fig:mL_mT_paths}}
\end{figure}
\end{frame}

\begin{frame}{Relating magnetizations}
\begin{equation}
    m_{t} =  \frac{3 \alpha m_{\ell}}{N - 2} + \frac{3\alpha\big[N(1-N)+4\alpha(N-\alpha)\big]}{N(N-1)(N-2)} + 1.
\label{eq:mT_of_mL}
\end{equation}
The equation for path $A$ is
\begin{equation}
    \begin{split}
        f(m_\ell) = 8N^2 - 9(N-1)(Nm_\ell -2)- (3m_\ell(N-1) + 3+N)^{3 / 2}
    \end{split}
    \label{eq:rs}
\end{equation}
\end{frame}

\subsection{More data results}

\begin{frame}{More data results}
\begin{figure}[h]
    \centering
	\includegraphics[height=0.5\textwidth]{ts_ML}
    \caption{Fluctuation of the link magnetization. Top panel: raw series; lower panel: average every 50 consecutive data points. Red lines represent the mean value for each plot.}
    \label{fig:ts1}
\end{figure}
\end{frame}

\begin{frame}{More data results}
\begin{figure}[h]
    \centering
    \includegraphics[height=0.5\textwidth]{ts_MT}
    \caption{Fluctuation of the triangle magnetization. Top panel: raw series; lower panel: average every 50 consecutive data points. Red lines represent the mean value for each plot.}
    \label{fig:ts2}
\end{figure}
\end{frame}
%
%\begin{frame}{}
%\begin{figure}[h]
%    \centering
%    \includegraphics[height=0.5\textwidth]{ts_SL}
%    \caption{Fluctuation of the link entropy (computed with $\k=1$). Top panel: raw series; lower panel: average every 50 consecutive data points. Red lines represent the mean value for each plot.}
%    \label{fig:ts3}
%\end{figure}
%}
%\begin{frame}{}
%\begin{figure}[h]
%    \centering
%    \includegraphics[height=0.5\textwidth]{ts_ST}
%    \caption{Fluctuation of the triangle magnetization (computed with $\k=1$). Top panel: raw series; lower panel: average every 50 consecutive data points. Red lines represent the mean value for each plot.}
%    \label{fig:ts4}
%\end{figure}
%
%\begin{frame}{}
%\begin{figure}[h]
%    \centering
%    \includegraphics[height=0.5\textwidth]{ts_T}
%    \caption{Fluctuation of the T (computed with $\k=1$). Top panel: raw series; lower panel: average every 50 consecutive data points. Red lines represent the mean value for each plot.}
%    \label{fig:ts5}
%\end{figure}
%}




